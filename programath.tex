% Basic stuff
\documentclass{article}
\usepackage[utf8]{inputenc}
\usepackage[a4paper, total={6.5in, 9.5in}]{geometry}
\usepackage[bookmarks, hidelinks, unicode]{hyperref}
\usepackage[]{amsmath,amssymb}
\usepackage{stmaryrd}
\usepackage{tikz}
\usepackage{lmodern}
\usepackage{soul}
\usepackage{float}
\usepackage{minted}
\usepackage{multicol}
%\usepackage{unicode-math}

% Packages configuration
\usetikzlibrary{shapes.arrows, angles, quotes}
\renewcommand{\arraystretch}{1.4}
\restylefloat{table}

% Shortcut commands
\newcommand{\im}{\text{Im}\,}
\newcommand{\re}{\text{Re}\,}
\newcommand{\img}{\text{Img}\,}
\newcommand{\R}{{\mathbb R}}
\renewcommand{\C}{{\mathbb C}}
\newcommand{\N}{{\mathbb N}}
\newcommand{\Z}{{\mathbb Z}}
\newcommand{\Q}{{\mathbb Q}}
\renewcommand{\U}{{\mathbb U}}
\newcommand{\cC}{{\mathcal C}}
\newcommand{\cD}{{\mathcal D}}
\newcommand{\cF}{{\mathcal F}}
\newcommand{\cotan}{\operatorname{cotan}}
\newcommand{\conj}[1]{\overline{#1}}
\newcommand{\Aff}{\text{Aff}}
\newcommand{\twoRows}[1]{\multirow{2}{*}{#1}}
\newcommand{\threeRows}[1]{\multirow{3}{*}{#1}}
\newcommand{\twoCols}[1]{\multicolumn{2}{c|}{#1}}
\newcommand{\threeCols}[1]{\multicolumn{3}{|c|}{#1}}
\newcommand{\twoColsNB}[1]{\multicolumn{2}{c}{#1}}
\newcommand{\goesto}[2]{\xrightarrow[#1\:\to\:#2]{}}
\newcommand{\liminfty}{\lim_{x\to+\infty}}
\newcommand{\limminfty}{\lim_{x\to-\infty}}
\newcommand{\limzero}{\lim_{x\to0}}
\newcommand{ \const}{\text{cste}}
\newcommand{\et}{\:\text{et}\:}
\newcommand{\ou}{\:\text{ou}\:}
\newcommand{\placeholder}{\diamond}
\newcommand{\mediateur}{\:\text{med}\:}
\newcommand{\milieu}{\:\text{mil}\:}
\newcommand{\vect}[1]{\overrightarrow{#1}}
\newcommand{\point}[2]{(#1;\;#2)}
\newcommand{\spacepoint}[3]{\begin{pmatrix}#1\\#2\\#3\end{pmatrix}}
\newcommand{\sh}{\operatorname{sh}}
\newcommand{\ch}{\operatorname{ch}}
\renewcommand{\th}{\operatorname{th}}
\newcommand{\id}{\operatorname{id}}
\renewcommand{\cong}{\equiv}
\newcommand{\converges}[2]{\xrightarrow[{#1\to #2}]{}}
\newcommand{\convergedby}[2]{\xleftarrow[{#1\to #2}]{}}

\newcommand{\singlecomparison}[2]{%
\begin{table}[h]
	\centering
	\begin{tabular}{p{7.5cm}|p{7.5cm}}
Python & Math \\\hline
\begin{minipage}{0.5\linewidth}
\begin{minted}{python}
#1
\end{minted}
\end{minipage}
       & #2
\end{tabular}
\end{table}
}

\newenvironment{python}{%
\begin{minipage}{0.5\linewidth}
\begin{minted}{python}
}{%
\end{minted}
\end{minipage}
}

% Document
\begin{document}

\section*{\LARGE Programath}
\tableofcontents

\newpage
\section{\texttt{=} vs \texttt{==}}
En maths, on note indifféremment le = de la \textbf{déclaration} et le = de l'\textbf{hypothèse}.

\begin{itemize}
	\item Ce que j'appelle le \emph{= de la déclaration}, c'est celui qu'on utilise pour poser une variable: "Soit $a = 45$".
	\item Ce que j'appelle le \emph{= de l'hypothèse}, c'est le = dont on n'est pas sûr, celui que l'on veut prouver ou réfuter: "Supposons $b = c+d$".
\end{itemize}

En programmation, ce que vous dites est interprété par une machine, qui ne peut pas déduire cette différence cruciale toute seule.
On est donc obligé de noter les "deux =" différement:

\begin{table}[h]
	\centering
	\begin{tabular}{p{7.5cm}|p{7.5cm}}
Python & Math \\\hline
\begin{minipage}{0.5\linewidth}
\begin{minted}{python}
a = 45
\end{minted}
\end{minipage}
       & Soit $a = 45$ \\
\begin{minipage}{0.5\linewidth}
\begin{minted}{python}
a == 45
\end{minted} 
\end{minipage} & $a = 45$ \\

\begin{minipage}{0.5\linewidth}
\begin{minted}{python}
P = a == 45
\end{minted} 
\end{minipage} & Soit $P$ la propriété "$a = 45$" \\
\end{tabular}
\end{table}

Comme en programmation on pose (ie on déclare) plus que l'on ne teste, le "=" tout simple sert de "= de la déclaration".

\section{Syntaxe de base}
+, $\times$ et tout ça.
\begin{table}[h]
	\centering
	\begin{tabular}{p{7.5cm}|p{7.5cm}}
\begin{minipage}{0.5\linewidth}
\begin{minted}{python}
(1 + 2) * 3 - 4**6
\end{minted}
\end{minipage}
       & $(1+2)\cdot 3 - 4^6$ \\
\begin{minipage}{0.5\linewidth}
\begin{minted}{python}
a//b + a/b
\end{minted} 
\end{minipage} & $\displaystyle \left\lfloor \frac{a}{b} \right\rfloor + \frac{a}{c}$ \\

\begin{minipage}{0.5\linewidth}
\begin{minted}{python}
r = a%b
\end{minted}
\end{minipage}
       & Posons $r$ tel que $a = \left\lfloor \frac{a}{b} \right\rfloor b + r$ \\
\hline
\begin{minipage}{0.5\linewidth}
\begin{minted}{python}
from math import sqrt 
sqrt(5) 
\end{minted}
\end{minipage}
       & $ \sqrt{5}  $ \\
\end{tabular}
\end{table}

\section{Ensembles, intervalles}


\begin{table}[h]
	\centering
	\begin{tabular}{p{7.5cm}|p{7.5cm}}
\begin{minipage}{0.5\linewidth}
\begin{minted}{python}
range(a, b)
\end{minted}
\end{minipage}
       & $\llbracket a, b \llbracket$ \\

\begin{minipage}{0.5\linewidth}
\begin{minted}{python}
{ 2*a for a in A }
\end{minted}
\end{minipage}
       & $ \{2a,\ a \in A\}$ \\
\begin{minipage}{0.5\linewidth}
\begin{minted}{python}
{ a for a in A if a**2 == a/2 }
\end{minted}
\end{minipage}
       & $ \{a \in A,\ a^2 = \frac{a}{2}\}$ \\
\multicolumn{2}{c}{Une version qui mélange les deux} \\

\begin{minipage}{0.5\linewidth}
\begin{minted}{python}
{ f(a) for a in A if P(a) }
\end{minted}
\end{minipage}
       & $ \{f(a),\ a \in A,\ P(a)\}$ \emph{(pas vraiment légal mais on l'a utilisé une fois)}  \\
\end{tabular}
\end{table}

\subsection{\texttt{[]} vs \texttt{\{\}}}




\end{document}

